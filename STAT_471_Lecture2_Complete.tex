% Options for packages loaded elsewhere
\PassOptionsToPackage{unicode}{hyperref}
\PassOptionsToPackage{hyphens}{url}
%
\documentclass[
]{article}
\usepackage{amsmath,amssymb}
\usepackage{iftex}
\ifPDFTeX
  \usepackage[T1]{fontenc}
  \usepackage[utf8]{inputenc}
  \usepackage{textcomp} % provide euro and other symbols
\else % if luatex or xetex
  \usepackage{unicode-math} % this also loads fontspec
  \defaultfontfeatures{Scale=MatchLowercase}
  \defaultfontfeatures[\rmfamily]{Ligatures=TeX,Scale=1}
\fi
\usepackage{lmodern}
\ifPDFTeX\else
  % xetex/luatex font selection
\fi
% Use upquote if available, for straight quotes in verbatim environments
\IfFileExists{upquote.sty}{\usepackage{upquote}}{}
\IfFileExists{microtype.sty}{% use microtype if available
  \usepackage[]{microtype}
  \UseMicrotypeSet[protrusion]{basicmath} % disable protrusion for tt fonts
}{}
\makeatletter
\@ifundefined{KOMAClassName}{% if non-KOMA class
  \IfFileExists{parskip.sty}{%
    \usepackage{parskip}
  }{% else
    \setlength{\parindent}{0pt}
    \setlength{\parskip}{6pt plus 2pt minus 1pt}}
}{% if KOMA class
  \KOMAoptions{parskip=half}}
\makeatother
\usepackage{xcolor}
\usepackage[margin=1in]{geometry}
\usepackage{color}
\usepackage{fancyvrb}
\newcommand{\VerbBar}{|}
\newcommand{\VERB}{\Verb[commandchars=\\\{\}]}
\DefineVerbatimEnvironment{Highlighting}{Verbatim}{commandchars=\\\{\}}
% Add ',fontsize=\small' for more characters per line
\usepackage{framed}
\definecolor{shadecolor}{RGB}{248,248,248}
\newenvironment{Shaded}{\begin{snugshade}}{\end{snugshade}}
\newcommand{\AlertTok}[1]{\textcolor[rgb]{0.94,0.16,0.16}{#1}}
\newcommand{\AnnotationTok}[1]{\textcolor[rgb]{0.56,0.35,0.01}{\textbf{\textit{#1}}}}
\newcommand{\AttributeTok}[1]{\textcolor[rgb]{0.13,0.29,0.53}{#1}}
\newcommand{\BaseNTok}[1]{\textcolor[rgb]{0.00,0.00,0.81}{#1}}
\newcommand{\BuiltInTok}[1]{#1}
\newcommand{\CharTok}[1]{\textcolor[rgb]{0.31,0.60,0.02}{#1}}
\newcommand{\CommentTok}[1]{\textcolor[rgb]{0.56,0.35,0.01}{\textit{#1}}}
\newcommand{\CommentVarTok}[1]{\textcolor[rgb]{0.56,0.35,0.01}{\textbf{\textit{#1}}}}
\newcommand{\ConstantTok}[1]{\textcolor[rgb]{0.56,0.35,0.01}{#1}}
\newcommand{\ControlFlowTok}[1]{\textcolor[rgb]{0.13,0.29,0.53}{\textbf{#1}}}
\newcommand{\DataTypeTok}[1]{\textcolor[rgb]{0.13,0.29,0.53}{#1}}
\newcommand{\DecValTok}[1]{\textcolor[rgb]{0.00,0.00,0.81}{#1}}
\newcommand{\DocumentationTok}[1]{\textcolor[rgb]{0.56,0.35,0.01}{\textbf{\textit{#1}}}}
\newcommand{\ErrorTok}[1]{\textcolor[rgb]{0.64,0.00,0.00}{\textbf{#1}}}
\newcommand{\ExtensionTok}[1]{#1}
\newcommand{\FloatTok}[1]{\textcolor[rgb]{0.00,0.00,0.81}{#1}}
\newcommand{\FunctionTok}[1]{\textcolor[rgb]{0.13,0.29,0.53}{\textbf{#1}}}
\newcommand{\ImportTok}[1]{#1}
\newcommand{\InformationTok}[1]{\textcolor[rgb]{0.56,0.35,0.01}{\textbf{\textit{#1}}}}
\newcommand{\KeywordTok}[1]{\textcolor[rgb]{0.13,0.29,0.53}{\textbf{#1}}}
\newcommand{\NormalTok}[1]{#1}
\newcommand{\OperatorTok}[1]{\textcolor[rgb]{0.81,0.36,0.00}{\textbf{#1}}}
\newcommand{\OtherTok}[1]{\textcolor[rgb]{0.56,0.35,0.01}{#1}}
\newcommand{\PreprocessorTok}[1]{\textcolor[rgb]{0.56,0.35,0.01}{\textit{#1}}}
\newcommand{\RegionMarkerTok}[1]{#1}
\newcommand{\SpecialCharTok}[1]{\textcolor[rgb]{0.81,0.36,0.00}{\textbf{#1}}}
\newcommand{\SpecialStringTok}[1]{\textcolor[rgb]{0.31,0.60,0.02}{#1}}
\newcommand{\StringTok}[1]{\textcolor[rgb]{0.31,0.60,0.02}{#1}}
\newcommand{\VariableTok}[1]{\textcolor[rgb]{0.00,0.00,0.00}{#1}}
\newcommand{\VerbatimStringTok}[1]{\textcolor[rgb]{0.31,0.60,0.02}{#1}}
\newcommand{\WarningTok}[1]{\textcolor[rgb]{0.56,0.35,0.01}{\textbf{\textit{#1}}}}
\usepackage{longtable,booktabs,array}
\usepackage{calc} % for calculating minipage widths
% Correct order of tables after \paragraph or \subparagraph
\usepackage{etoolbox}
\makeatletter
\patchcmd\longtable{\par}{\if@noskipsec\mbox{}\fi\par}{}{}
\makeatother
% Allow footnotes in longtable head/foot
\IfFileExists{footnotehyper.sty}{\usepackage{footnotehyper}}{\usepackage{footnote}}
\makesavenoteenv{longtable}
\usepackage{graphicx}
\makeatletter
\newsavebox\pandoc@box
\newcommand*\pandocbounded[1]{% scales image to fit in text height/width
  \sbox\pandoc@box{#1}%
  \Gscale@div\@tempa{\textheight}{\dimexpr\ht\pandoc@box+\dp\pandoc@box\relax}%
  \Gscale@div\@tempb{\linewidth}{\wd\pandoc@box}%
  \ifdim\@tempb\p@<\@tempa\p@\let\@tempa\@tempb\fi% select the smaller of both
  \ifdim\@tempa\p@<\p@\scalebox{\@tempa}{\usebox\pandoc@box}%
  \else\usebox{\pandoc@box}%
  \fi%
}
% Set default figure placement to htbp
\def\fps@figure{htbp}
\makeatother
\setlength{\emergencystretch}{3em} % prevent overfull lines
\providecommand{\tightlist}{%
  \setlength{\itemsep}{0pt}\setlength{\parskip}{0pt}}
\setcounter{secnumdepth}{-\maxdimen} % remove section numbering
\usepackage{bookmark}
\IfFileExists{xurl.sty}{\usepackage{xurl}}{} % add URL line breaks if available
\urlstyle{same}
\hypersetup{
  pdftitle={R Notebook},
  hidelinks,
  pdfcreator={LaTeX via pandoc}}

\title{R Notebook}
\author{}
\date{\vspace{-2.5em}}

\begin{document}
\maketitle

\subsubsection{\texorpdfstring{\textbf{For Loops, While Loops, and
Custom
Functions}}{For Loops, While Loops, and Custom Functions}}\label{for-loops-while-loops-and-custom-functions}

\subsubsection{\texorpdfstring{\textbf{For
Loops}}{For Loops}}\label{for-loops}

\begin{itemize}
\tightlist
\item
  In math and stats, we have dealt with summations and product
  iterations in the past.
\item
  We can express sums, products, and any iterative procedures as a loop.
\item
  With sums, addition is ``iterated'' each time, from a starting index
  to an ending index.
\item
  You can think of this as how for and while loops work.
\end{itemize}

\begin{Shaded}
\begin{Highlighting}[]
\NormalTok{sum }\OtherTok{=} \DecValTok{0}
\ControlFlowTok{for}\NormalTok{ (i }\ControlFlowTok{in} \DecValTok{1}\SpecialCharTok{:}\DecValTok{10}\NormalTok{) \{}
\NormalTok{  sum }\OtherTok{=}\NormalTok{ sum }\SpecialCharTok{+}\NormalTok{ i}
\NormalTok{\}}
\NormalTok{sum}
\end{Highlighting}
\end{Shaded}

\begin{verbatim}
## [1] 55
\end{verbatim}

\subsubsection{\texorpdfstring{\textbf{Example: Sorting a Vector without
Built-In
Functions}}{Example: Sorting a Vector without Built-In Functions}}\label{example-sorting-a-vector-without-built-in-functions}

The idea: To sort a vector, we want to check each pair of successive
elements in a vector and compare the greatest value.

The algorithm:

\begin{enumerate}
\def\labelenumi{\arabic{enumi}.}
\item
  Write a for loop to iterate from the first element to the last ``nth''
  element.
\item
  Create another for loop to iterate from the 0th element to the each
  element that'll be compared next (n-i)th element.
\item
  Create an if-else condition block. If the current element is greater
  than the next element in the vector, swap the elements.
\end{enumerate}

This sorting algorithm is called the ``bubble sort algorithm''. Here, it
sorts the vector in ascending order, but you can switch the conditional
sign to sort by descending order.

\begin{Shaded}
\begin{Highlighting}[]
\NormalTok{vec }\OtherTok{=} \FunctionTok{c}\NormalTok{(}\DecValTok{76}\NormalTok{, }\DecValTok{23}\NormalTok{, }\DecValTok{45}\NormalTok{, }\DecValTok{12}\NormalTok{, }\DecValTok{54}\NormalTok{, }\DecValTok{9}\NormalTok{)}
\NormalTok{n }\OtherTok{=} \FunctionTok{length}\NormalTok{(vec)}

\ControlFlowTok{for}\NormalTok{ (i }\ControlFlowTok{in} \DecValTok{1}\SpecialCharTok{:}\NormalTok{(n}\DecValTok{{-}1}\NormalTok{)) \{}
  \ControlFlowTok{for}\NormalTok{ (j }\ControlFlowTok{in} \DecValTok{1}\SpecialCharTok{:}\NormalTok{(n}\SpecialCharTok{{-}}\NormalTok{i)) \{}
    \ControlFlowTok{if}\NormalTok{ (vec[j] }\SpecialCharTok{\textgreater{}}\NormalTok{ vec[j}\SpecialCharTok{+}\DecValTok{1}\NormalTok{]) \{}
\NormalTok{      item }\OtherTok{=}\NormalTok{ vec[j]}
\NormalTok{      vec[j] }\OtherTok{=}\NormalTok{ vec[j}\SpecialCharTok{+}\DecValTok{1}\NormalTok{]}
\NormalTok{      vec[j}\SpecialCharTok{+}\DecValTok{1}\NormalTok{] }\OtherTok{=}\NormalTok{ item}
\NormalTok{    \}}
\NormalTok{  \}}
\NormalTok{\}}
\NormalTok{vec}
\end{Highlighting}
\end{Shaded}

\begin{verbatim}
## [1]  9 12 23 45 54 76
\end{verbatim}

\subsubsection{\texorpdfstring{\textbf{While
Loops}}{While Loops}}\label{while-loops}

While loops are iterative as well, however they're written differently
and the logic is dependent on the conditions you define.

\begin{itemize}
\tightlist
\item
  These conditions are controlled with logical values like TRUE or
  FALSE.
\item
  For loops automatically end once the maximum number of iterations are
  reached, while loops may run forever if not ended with a ``flag''
  condition.
\end{itemize}

\subsubsection{\texorpdfstring{\textbf{Example: Printing Numbers Using a
While
Loop}}{Example: Printing Numbers Using a While Loop}}\label{example-printing-numbers-using-a-while-loop}

\begin{Shaded}
\begin{Highlighting}[]
\NormalTok{num }\OtherTok{=} \DecValTok{1}

\ControlFlowTok{while}\NormalTok{ (num }\SpecialCharTok{\textless{}} \DecValTok{5}\NormalTok{) \{}
  \FunctionTok{print}\NormalTok{(num)}
\NormalTok{  num }\OtherTok{=}\NormalTok{ num }\SpecialCharTok{+} \DecValTok{1}
\NormalTok{\}}
\end{Highlighting}
\end{Shaded}

\begin{verbatim}
## [1] 1
## [1] 2
## [1] 3
## [1] 4
\end{verbatim}

\subsubsection{\texorpdfstring{\textbf{Example: Calculating Sum of
Elements in a
Vector}}{Example: Calculating Sum of Elements in a Vector}}\label{example-calculating-sum-of-elements-in-a-vector}

\begin{Shaded}
\begin{Highlighting}[]
\NormalTok{nums }\OtherTok{=} \FunctionTok{c}\NormalTok{(}\DecValTok{23}\NormalTok{, }\DecValTok{99}\NormalTok{, }\DecValTok{101}\NormalTok{, }\DecValTok{62}\NormalTok{, }\DecValTok{88}\NormalTok{)}
\NormalTok{total }\OtherTok{=} \DecValTok{0}
\NormalTok{n }\OtherTok{=} \FunctionTok{length}\NormalTok{(nums)}
\ControlFlowTok{for}\NormalTok{ (i }\ControlFlowTok{in} \DecValTok{1}\SpecialCharTok{:}\NormalTok{n) \{}
\NormalTok{  total }\OtherTok{=}\NormalTok{ total }\SpecialCharTok{+}\NormalTok{ nums[i]}
\NormalTok{\}}
\NormalTok{total}
\end{Highlighting}
\end{Shaded}

\begin{verbatim}
## [1] 373
\end{verbatim}

\subsubsection{\texorpdfstring{\textbf{Example: Calculating Accuracy of
a Machine Learning
Model}}{Example: Calculating Accuracy of a Machine Learning Model}}\label{example-calculating-accuracy-of-a-machine-learning-model}

Suppose you run your machine learning model on some training data to
classify if patients have pneumonia (yes=1) or not (no=0). You want to
calculate the accuracy of your model by comparing your predictions with
the known data you have.

\begin{itemize}
\tightlist
\item
  Our predictions are stored in the variable ``preds'' and our test data
  values are stored in the variable ``test\_vals''.
\end{itemize}

\begin{Shaded}
\begin{Highlighting}[]
\NormalTok{preds }\OtherTok{=} \FunctionTok{c}\NormalTok{(}\DecValTok{0}\NormalTok{, }\DecValTok{1}\NormalTok{, }\DecValTok{1}\NormalTok{, }\DecValTok{0}\NormalTok{, }\DecValTok{1}\NormalTok{, }\DecValTok{0}\NormalTok{, }\DecValTok{1}\NormalTok{, }\DecValTok{0}\NormalTok{, }\DecValTok{0}\NormalTok{, }\DecValTok{1}\NormalTok{)}
\NormalTok{test\_vals }\OtherTok{=} \FunctionTok{c}\NormalTok{(}\DecValTok{0}\NormalTok{, }\DecValTok{0}\NormalTok{, }\DecValTok{1}\NormalTok{, }\DecValTok{1}\NormalTok{, }\DecValTok{1}\NormalTok{, }\DecValTok{0}\NormalTok{, }\DecValTok{1}\NormalTok{, }\DecValTok{1}\NormalTok{, }\DecValTok{0}\NormalTok{, }\DecValTok{1}\NormalTok{)}
\end{Highlighting}
\end{Shaded}

To compute the accuracy of this model, we count the number of matches in
each vector and divide by the total observations we have. We'll write
some R code to compute the accuracy of our model and round it to two
decimal places.

\begin{Shaded}
\begin{Highlighting}[]
\NormalTok{acc\_count }\OtherTok{=} \DecValTok{0}

\CommentTok{\# Vectors are the same size, so you can use either pred\_len or test\_len here.}

\NormalTok{pred\_len }\OtherTok{=} \FunctionTok{length}\NormalTok{(preds)}
\NormalTok{test\_len }\OtherTok{=} \FunctionTok{length}\NormalTok{(test\_vals)}
\ControlFlowTok{for}\NormalTok{ (i }\ControlFlowTok{in} \DecValTok{1}\SpecialCharTok{:}\NormalTok{pred\_len) \{}
  \ControlFlowTok{if}\NormalTok{ (preds[i] }\SpecialCharTok{==}\NormalTok{ test\_vals[i]) \{}
\NormalTok{    acc\_count }\OtherTok{=}\NormalTok{ acc\_count }\SpecialCharTok{+} \DecValTok{1}
\NormalTok{  \}}
\NormalTok{\}}
\NormalTok{acc\_tot }\OtherTok{=}\NormalTok{ acc\_count }\SpecialCharTok{/}\NormalTok{ pred\_len}
\NormalTok{acc\_tot }\OtherTok{=} \FunctionTok{round}\NormalTok{(acc\_tot, }\DecValTok{2}\NormalTok{)}
\NormalTok{acc\_tot}
\end{Highlighting}
\end{Shaded}

\begin{verbatim}
## [1] 0.7
\end{verbatim}

\subsubsection{\texorpdfstring{\textbf{Functions}}{Functions}}\label{functions}

\begin{itemize}
\tightlist
\item
  Functions can organize multiple lines of code if you're handling
  multiple sub-tasks within a single task.
\item
  Alongside using built-in functions in R, you can write your own custom
  functions.
\end{itemize}

\subsubsection{\texorpdfstring{\textbf{Example: Writing a Function to
Calculate the Mean and Variance of a
Vector}}{Example: Writing a Function to Calculate the Mean and Variance of a Vector}}\label{example-writing-a-function-to-calculate-the-mean-and-variance-of-a-vector}

For this example, we'll use the mean and variance built-in functions.
Let's also round our outputs to two decimal places.

\begin{Shaded}
\begin{Highlighting}[]
\NormalTok{vec }\OtherTok{=} \FunctionTok{c}\NormalTok{(}\DecValTok{76}\NormalTok{, }\DecValTok{23}\NormalTok{, }\DecValTok{45}\NormalTok{, }\DecValTok{12}\NormalTok{, }\DecValTok{54}\NormalTok{, }\DecValTok{9}\NormalTok{)}
\NormalTok{stats\_fun }\OtherTok{=} \ControlFlowTok{function}\NormalTok{(x) \{}
\NormalTok{mean\_vec }\OtherTok{=} \FunctionTok{round}\NormalTok{(}\FunctionTok{mean}\NormalTok{(x), }\DecValTok{2}\NormalTok{)}
\NormalTok{var\_vec }\OtherTok{=} \FunctionTok{round}\NormalTok{(}\FunctionTok{var}\NormalTok{(x), }\DecValTok{2}\NormalTok{)}
\FunctionTok{print}\NormalTok{(mean\_vec)}
\FunctionTok{print}\NormalTok{(var\_vec)}
\NormalTok{\}}
\FunctionTok{stats\_fun}\NormalTok{(vec)}
\end{Highlighting}
\end{Shaded}

\begin{verbatim}
## [1] 36.5
## [1] 695.5
\end{verbatim}

\subsubsection{\texorpdfstring{\textbf{Example: Writing a Function to
Standardize a
Vector}}{Example: Writing a Function to Standardize a Vector}}\label{example-writing-a-function-to-standardize-a-vector}

Recall that when we standardize data, we want to scale each data value
as so:

(x - min)/(max - min)

\begin{Shaded}
\begin{Highlighting}[]
\NormalTok{vec }\OtherTok{=} \FunctionTok{c}\NormalTok{(}\DecValTok{76}\NormalTok{, }\DecValTok{23}\NormalTok{, }\DecValTok{45}\NormalTok{, }\DecValTok{12}\NormalTok{, }\DecValTok{54}\NormalTok{, }\DecValTok{9}\NormalTok{)}
\NormalTok{standardization }\OtherTok{=} \ControlFlowTok{function}\NormalTok{(x) \{}
\NormalTok{n }\OtherTok{=} \FunctionTok{length}\NormalTok{(x)}
\ControlFlowTok{for}\NormalTok{ (i }\ControlFlowTok{in} \DecValTok{1}\SpecialCharTok{:}\NormalTok{n) \{}
\NormalTok{  x[i] }\OtherTok{=} \FunctionTok{round}\NormalTok{((x[i] }\SpecialCharTok{{-}} \FunctionTok{min}\NormalTok{(x)) }\SpecialCharTok{/}\NormalTok{ (}\FunctionTok{max}\NormalTok{(x) }\SpecialCharTok{{-}} \FunctionTok{min}\NormalTok{(x)), }\DecValTok{2}\NormalTok{)}
\NormalTok{\}}
\FunctionTok{print}\NormalTok{(x)}
\NormalTok{\}}
\FunctionTok{standardization}\NormalTok{(vec)}
\end{Highlighting}
\end{Shaded}

\begin{verbatim}
## [1] 1.00 0.42 0.83 0.22 1.00 1.00
\end{verbatim}

\subsubsection{Data Frames vs.~Tibbles}\label{data-frames-vs.-tibbles}

\paragraph{R packages}\label{r-packages}

A common point of confusion for new R users is understanding R packages.
R packages extend the functionality of R by providing additional
functions, datasets, and documentation. These packages are developed by
a global community and can be downloaded for free.

\textbf{A helpful analogy:} Think of R as a brand-new smartphone. It has
basic functionality out of the box, but to do more, you'll want to
download apps. R packages are like these apps.

For example, say you've just bought a new phone and want to share a
photo on \emph{Instagram} You need to:

\begin{enumerate}
\def\labelenumi{\arabic{enumi}.}
\tightlist
\item
  \textbf{Install} the app.
\item
  \textbf{Open} the app.
\end{enumerate}

Similarly, to use an R package, you must:

\begin{enumerate}
\def\labelenumi{\arabic{enumi}.}
\tightlist
\item
  \textbf{Install} the package.
\item
  \textbf{Load} the package.
\end{enumerate}

Most packages are not installed by default when you install R and
RStudio. If you want to use a package for the first time, you must
install it first.

Likewise, packages are not automatically loaded when you start RStudio.
You'll need to load each package you want to use every time you start a
new R session.

\paragraph{Installing a Package}\label{installing-a-package}

There are two common ways to install a package in R:

\textbf{Option 1: Using the RStudio Interface}

\begin{enumerate}
\def\labelenumi{\arabic{enumi}.}
\tightlist
\item
  Click \textbf{Tools → Install Packages}
\item
  Type the name of the package into the text box.
\item
  Click \textbf{Install}.
\end{enumerate}

\textbf{Option 2: Using Code in the Console}

You can also install a package by running the following command in the
console:

\begin{Shaded}
\begin{Highlighting}[]
\CommentTok{\#install.packages("ggplot2")}
\end{Highlighting}
\end{Shaded}

Make sure to include quotation marks around the package name.

\paragraph{Loading a Package}\label{loading-a-package}

Once you've installed a package, you'll need to load it into your
current R session using library() before you can use it. lp

\begin{Shaded}
\begin{Highlighting}[]
\FunctionTok{library}\NormalTok{(ggplot2)}
\end{Highlighting}
\end{Shaded}

\begin{verbatim}
## Warning: package 'ggplot2' was built under R version 4.3.3
\end{verbatim}

\begin{Shaded}
\begin{Highlighting}[]
\CommentTok{\# or}
\CommentTok{\#library("ggplot2")}
\end{Highlighting}
\end{Shaded}

If the console returns a blinking cursor with no error, the package
loaded successfully.

If you see:
\texttt{Error\ in\ library(ggplot2)\ :\ there\ is\ no\ package\ called\ ‘ggplot2’}

It means the package was not installed successfully.

\paragraph{Exercise 1}\label{exercise-1}

\begin{enumerate}
\def\labelenumi{\arabic{enumi}.}
\item
  Install the following packages:

  \begin{itemize}
  \item
    \texttt{dplyr}: Functions for manipulating and transforming data
    frames.
  \item
    \texttt{nycflights13}: Dataset of all flights departing from NYC
    airports (JFK, LGA, EWR) in 2013.
  \item
    \texttt{knitr}: Functions for generating reports and tables.
  \end{itemize}
\item
  Load each of these packages using \texttt{library()}.
\end{enumerate}

\begin{Shaded}
\begin{Highlighting}[]
\CommentTok{\#install.packages(c("dplyr", "nycflights13", "knitr"))}

\FunctionTok{library}\NormalTok{(dplyr)}
\end{Highlighting}
\end{Shaded}

\begin{verbatim}
## Warning: package 'dplyr' was built under R version 4.3.3
\end{verbatim}

\begin{verbatim}
## 
## Attaching package: 'dplyr'
\end{verbatim}

\begin{verbatim}
## The following objects are masked from 'package:stats':
## 
##     filter, lag
\end{verbatim}

\begin{verbatim}
## The following objects are masked from 'package:base':
## 
##     intersect, setdiff, setequal, union
\end{verbatim}

\begin{Shaded}
\begin{Highlighting}[]
\FunctionTok{library}\NormalTok{(nycflights13)}
\end{Highlighting}
\end{Shaded}

\begin{verbatim}
## Warning: package 'nycflights13' was built under R version 4.3.3
\end{verbatim}

\begin{Shaded}
\begin{Highlighting}[]
\FunctionTok{library}\NormalTok{(knitr)}
\end{Highlighting}
\end{Shaded}

\begin{verbatim}
## Warning: package 'knitr' was built under R version 4.3.3
\end{verbatim}

\paragraph{The Tidyverse Package}\label{the-tidyverse-package}

The \texttt{tidyverse} is a collection of R packages designed for
importing, cleaning, transforming, visualizing, and modeling data.

Some core tidyverse packages include:

\begin{itemize}
\tightlist
\item
  \texttt{ggplot2}: data visualization
\item
  \texttt{dplyr}: data manipulation
\item
  \texttt{tidyr}: reshaping and tidying data
\item
  \texttt{readr}: importing data
\item
  \texttt{tibble}: modern data frames
\item
  \texttt{stringr}: working with strings
\item
  \texttt{forcats}: working with factors
\end{itemize}

\begin{Shaded}
\begin{Highlighting}[]
\CommentTok{\#install.packages("tidyverse")}
\FunctionTok{library}\NormalTok{(tidyverse)}
\end{Highlighting}
\end{Shaded}

\begin{verbatim}
## Warning: package 'tidyverse' was built under R version 4.3.3
\end{verbatim}

\begin{verbatim}
## Warning: package 'stringr' was built under R version 4.3.3
\end{verbatim}

\begin{verbatim}
## Warning: package 'forcats' was built under R version 4.3.3
\end{verbatim}

\begin{verbatim}
## Warning: package 'lubridate' was built under R version 4.3.3
\end{verbatim}

\begin{verbatim}
## -- Attaching core tidyverse packages ------------------------ tidyverse 2.0.0 --
## v forcats   1.0.0     v stringr   1.5.1
## v lubridate 1.9.3     v tibble    3.2.1
## v purrr     1.0.2     v tidyr     1.3.1
## v readr     2.1.5     
## -- Conflicts ------------------------------------------ tidyverse_conflicts() --
## x dplyr::filter() masks stats::filter()
## x dplyr::lag()    masks stats::lag()
## i Use the conflicted package (<http://conflicted.r-lib.org/>) to force all conflicts to become errors
\end{verbatim}

This will automatically load the core \texttt{tidyverse} packages.

\paragraph{Forgetting to Load a
Package}\label{forgetting-to-load-a-package}

You must load a package each time you start RStudio.

If you forget, you might see:
\texttt{Error:\ could\ not\ find\ function}

This means R doesn't know where the function is because the package
isn't loaded

\paragraph{\texorpdfstring{R's Traditional
\texttt{data.frame}}{R's Traditional data.frame}}\label{rs-traditional-data.frame}

A \textbf{data frame} is a list of variables (columns) with the same
number of rows, where each row has a unique row name.

For example, the \texttt{cars} data belongs to the class
\texttt{data.frame}.

\begin{Shaded}
\begin{Highlighting}[]
\CommentTok{\#data(cars)}

\CommentTok{\# check type}
\FunctionTok{class}\NormalTok{(cars)}
\end{Highlighting}
\end{Shaded}

\begin{verbatim}
## [1] "data.frame"
\end{verbatim}

\begin{Shaded}
\begin{Highlighting}[]
\CommentTok{\# column names }
\FunctionTok{colnames}\NormalTok{(cars)}
\end{Highlighting}
\end{Shaded}

\begin{verbatim}
## [1] "speed" "dist"
\end{verbatim}

\begin{Shaded}
\begin{Highlighting}[]
\CommentTok{\# row names}
\CommentTok{\#rownames(cars)}

\CommentTok{\# check dimensions (rows, columns)}
\FunctionTok{dim}\NormalTok{(cars)}
\end{Highlighting}
\end{Shaded}

\begin{verbatim}
## [1] 50  2
\end{verbatim}

\begin{Shaded}
\begin{Highlighting}[]
\CommentTok{\# first rows }
\FunctionTok{head}\NormalTok{(cars)}
\end{Highlighting}
\end{Shaded}

\begin{verbatim}
##   speed dist
## 1     4    2
## 2     4   10
## 3     7    4
## 4     7   22
## 5     8   16
## 6     9   10
\end{verbatim}

\begin{Shaded}
\begin{Highlighting}[]
\CommentTok{\# last rows}
\FunctionTok{tail}\NormalTok{(cars)}
\end{Highlighting}
\end{Shaded}

\begin{verbatim}
##    speed dist
## 45    23   54
## 46    24   70
## 47    24   92
## 48    24   93
## 49    24  120
## 50    25   85
\end{verbatim}

\paragraph{Tibbles}\label{tibbles}

\textbf{Tibbles} are data frames, but they tweak some older behaviors to
make them easier to use.

Two main differences from a classic data frame:

\begin{enumerate}
\def\labelenumi{\arabic{enumi}.}
\item
  \textbf{Printing:} Shows only the first 10 rows and all columns that
  fit on screen, along with each column's data type.
\item
  \textbf{Subsetting:} Behaves more consistently and avoids partial name
  matching.
\end{enumerate}

We can create a tibble using the \texttt{cars} dataset:

\begin{Shaded}
\begin{Highlighting}[]
\NormalTok{cars\_tibble }\OtherTok{\textless{}{-}} \FunctionTok{as\_tibble}\NormalTok{(cars)}
\NormalTok{cars\_tibble}
\end{Highlighting}
\end{Shaded}

\begin{verbatim}
## # A tibble: 50 x 2
##    speed  dist
##    <dbl> <dbl>
##  1     4     2
##  2     4    10
##  3     7     4
##  4     7    22
##  5     8    16
##  6     9    10
##  7    10    18
##  8    10    26
##  9    10    34
## 10    11    17
## # i 40 more rows
\end{verbatim}

\paragraph{Subsetting}\label{subsetting}

To extract a single column from a data frame, you can use\texttt{\$}or
\texttt{{[}{[}\ {]}{]}}:

\begin{Shaded}
\begin{Highlighting}[]
\CommentTok{\# extract by name}
\NormalTok{cars}\SpecialCharTok{$}\NormalTok{speed}
\end{Highlighting}
\end{Shaded}

\begin{verbatim}
##  [1]  4  4  7  7  8  9 10 10 10 11 11 12 12 12 12 13 13 13 13 14 14 14 14 15 15
## [26] 15 16 16 17 17 17 18 18 18 18 19 19 19 20 20 20 20 20 22 23 24 24 24 24 25
\end{verbatim}

\begin{Shaded}
\begin{Highlighting}[]
\CommentTok{\# extract by name}
\NormalTok{cars[[}\StringTok{"speed"}\NormalTok{]]}
\end{Highlighting}
\end{Shaded}

\begin{verbatim}
##  [1]  4  4  7  7  8  9 10 10 10 11 11 12 12 12 12 13 13 13 13 14 14 14 14 15 15
## [26] 15 16 16 17 17 17 18 18 18 18 19 19 19 20 20 20 20 20 22 23 24 24 24 24 25
\end{verbatim}

\begin{Shaded}
\begin{Highlighting}[]
\CommentTok{\# extract by position}
\NormalTok{cars[[}\DecValTok{1}\NormalTok{]]}
\end{Highlighting}
\end{Shaded}

\begin{verbatim}
##  [1]  4  4  7  7  8  9 10 10 10 11 11 12 12 12 12 13 13 13 13 14 14 14 14 15 15
## [26] 15 16 16 17 17 17 18 18 18 18 19 19 19 20 20 20 20 20 22 23 24 24 24 24 25
\end{verbatim}

The same methods work with tibbles:

\begin{Shaded}
\begin{Highlighting}[]
\NormalTok{cars\_tibble}\SpecialCharTok{$}\NormalTok{speed}
\end{Highlighting}
\end{Shaded}

\begin{verbatim}
##  [1]  4  4  7  7  8  9 10 10 10 11 11 12 12 12 12 13 13 13 13 14 14 14 14 15 15
## [26] 15 16 16 17 17 17 18 18 18 18 19 19 19 20 20 20 20 20 22 23 24 24 24 24 25
\end{verbatim}

\begin{Shaded}
\begin{Highlighting}[]
\NormalTok{cars\_tibble[[}\StringTok{"speed"}\NormalTok{]]}
\end{Highlighting}
\end{Shaded}

\begin{verbatim}
##  [1]  4  4  7  7  8  9 10 10 10 11 11 12 12 12 12 13 13 13 13 14 14 14 14 15 15
## [26] 15 16 16 17 17 17 18 18 18 18 19 19 19 20 20 20 20 20 22 23 24 24 24 24 25
\end{verbatim}

\begin{Shaded}
\begin{Highlighting}[]
\NormalTok{cars\_tibble[[}\DecValTok{1}\NormalTok{]]}
\end{Highlighting}
\end{Shaded}

\begin{verbatim}
##  [1]  4  4  7  7  8  9 10 10 10 11 11 12 12 12 12 13 13 13 13 14 14 14 14 15 15
## [26] 15 16 16 17 17 17 18 18 18 18 19 19 19 20 20 20 20 20 22 23 24 24 24 24 25
\end{verbatim}

You can also use the \textbf{pipe operator} with \texttt{pull()}:

\begin{Shaded}
\begin{Highlighting}[]
\NormalTok{cars\_tibble }\SpecialCharTok{|\textgreater{}} \FunctionTok{pull}\NormalTok{(speed)}
\end{Highlighting}
\end{Shaded}

\begin{verbatim}
##  [1]  4  4  7  7  8  9 10 10 10 11 11 12 12 12 12 13 13 13 13 14 14 14 14 15 15
## [26] 15 16 16 17 17 17 18 18 18 18 19 19 19 20 20 20 20 20 22 23 24 24 24 24 25
\end{verbatim}

\begin{Shaded}
\begin{Highlighting}[]
\CommentTok{\#cars\_tibble \%\textgreater{}\% pull(speed)}
\end{Highlighting}
\end{Shaded}

The pipe operator (\texttt{\textbar{}\textgreater{}} or
\texttt{\%\textgreater{}\%}) is a tool that let's you pass the result of
one function directly into the next function, making your code flow more
naturally.

\paragraph{Partial Name Matching}\label{partial-name-matching}

In a \textbf{data frame}, R will try to guess the column name if you
only type part of it.

\begin{Shaded}
\begin{Highlighting}[]
\CommentTok{\# data frame will match "spe" to "speed"}
\CommentTok{\#cars$spe}
\end{Highlighting}
\end{Shaded}

In a \textbf{tibble}, R will not guess.

\begin{Shaded}
\begin{Highlighting}[]
\CommentTok{\#tibble will give a warning}
\CommentTok{\#cars\_tibble$spe}
\end{Highlighting}
\end{Shaded}

\paragraph{\texorpdfstring{The \texttt{nycflights13}
Data}{The nycflights13 Data}}\label{the-nycflights13-data}

The \texttt{nycflights13} package contains five datasets stored in data
frames:

\begin{enumerate}
\def\labelenumi{\arabic{enumi}.}
\tightlist
\item
  \textbf{flights:} information on all 336,776 flights
\item
  \textbf{airlines:} airline names and IATA codes
\item
  \textbf{planes:} aircraft information
\item
  \textbf{weather:} hourly weather data for NYC airports
\item
  \textbf{airports:}airport names, codes, and locations
\end{enumerate}

We'll start by exploring the \texttt{flights} data frame.

\begin{Shaded}
\begin{Highlighting}[]
\FunctionTok{library}\NormalTok{(nycflights13)}
\NormalTok{flights}
\end{Highlighting}
\end{Shaded}

\begin{verbatim}
## # A tibble: 336,776 x 19
##     year month   day dep_time sched_dep_time dep_delay arr_time sched_arr_time
##    <int> <int> <int>    <int>          <int>     <dbl>    <int>          <int>
##  1  2013     1     1      517            515         2      830            819
##  2  2013     1     1      533            529         4      850            830
##  3  2013     1     1      542            540         2      923            850
##  4  2013     1     1      544            545        -1     1004           1022
##  5  2013     1     1      554            600        -6      812            837
##  6  2013     1     1      554            558        -4      740            728
##  7  2013     1     1      555            600        -5      913            854
##  8  2013     1     1      557            600        -3      709            723
##  9  2013     1     1      557            600        -3      838            846
## 10  2013     1     1      558            600        -2      753            745
## # i 336,766 more rows
## # i 11 more variables: arr_delay <dbl>, carrier <chr>, flight <int>,
## #   tailnum <chr>, origin <chr>, dest <chr>, air_time <dbl>, distance <dbl>,
## #   hour <dbl>, minute <dbl>, time_hour <dttm>
\end{verbatim}

\paragraph{Exploring Data Frames}\label{exploring-data-frames}

There are several ways to explore a data frame in R:

\begin{enumerate}
\def\labelenumi{\arabic{enumi}.}
\tightlist
\item
  \texttt{View()}: spreadsheet view of the dataset
\item
  \texttt{glimpse()}: compact overview of all variables and their data
  types.
\item
  \texttt{kable()}: nicely formatted tables for reports
\item
  \texttt{\$}: extraction operator
\end{enumerate}

\paragraph{Exercise 2}\label{exercise-2}

\begin{enumerate}
\def\labelenumi{\arabic{enumi}.}
\tightlist
\item
  Use \texttt{View()} on flights. What does one row represent?

  \begin{enumerate}
  \def\labelenumii{\alph{enumii}.}
  \tightlist
  \item
    Data on an airline
  \item
    Data on a flight ✅
  \item
    Data on an airport
  \item
    Data on multiple flights
  \end{enumerate}
\item
  Use \texttt{str()} on flights. How many character variables are there?
\end{enumerate}

\begin{Shaded}
\begin{Highlighting}[]
\FunctionTok{view}\NormalTok{(flights)}

\FunctionTok{str}\NormalTok{(flights)}
\end{Highlighting}
\end{Shaded}

\begin{verbatim}
## tibble [336,776 x 19] (S3: tbl_df/tbl/data.frame)
##  $ year          : int [1:336776] 2013 2013 2013 2013 2013 2013 2013 2013 2013 2013 ...
##  $ month         : int [1:336776] 1 1 1 1 1 1 1 1 1 1 ...
##  $ day           : int [1:336776] 1 1 1 1 1 1 1 1 1 1 ...
##  $ dep_time      : int [1:336776] 517 533 542 544 554 554 555 557 557 558 ...
##  $ sched_dep_time: int [1:336776] 515 529 540 545 600 558 600 600 600 600 ...
##  $ dep_delay     : num [1:336776] 2 4 2 -1 -6 -4 -5 -3 -3 -2 ...
##  $ arr_time      : int [1:336776] 830 850 923 1004 812 740 913 709 838 753 ...
##  $ sched_arr_time: int [1:336776] 819 830 850 1022 837 728 854 723 846 745 ...
##  $ arr_delay     : num [1:336776] 11 20 33 -18 -25 12 19 -14 -8 8 ...
##  $ carrier       : chr [1:336776] "UA" "UA" "AA" "B6" ...
##  $ flight        : int [1:336776] 1545 1714 1141 725 461 1696 507 5708 79 301 ...
##  $ tailnum       : chr [1:336776] "N14228" "N24211" "N619AA" "N804JB" ...
##  $ origin        : chr [1:336776] "EWR" "LGA" "JFK" "JFK" ...
##  $ dest          : chr [1:336776] "IAH" "IAH" "MIA" "BQN" ...
##  $ air_time      : num [1:336776] 227 227 160 183 116 150 158 53 140 138 ...
##  $ distance      : num [1:336776] 1400 1416 1089 1576 762 ...
##  $ hour          : num [1:336776] 5 5 5 5 6 5 6 6 6 6 ...
##  $ minute        : num [1:336776] 15 29 40 45 0 58 0 0 0 0 ...
##  $ time_hour     : POSIXct[1:336776], format: "2013-01-01 05:00:00" "2013-01-01 05:00:00" ...
\end{verbatim}

\begin{Shaded}
\begin{Highlighting}[]
\CommentTok{\#represent airline codes or airport codes }
\CommentTok{\#represent measurable quantities you can calculate with}
\end{Highlighting}
\end{Shaded}

\paragraph{\texorpdfstring{\texttt{glimpse()}}{glimpse()}}\label{glimpse}

The \texttt{glimpse()} function shows:

\begin{itemize}
\tightlist
\item
  The number of rows and columns.
\item
  Each variable's name and data type.
\item
  A preview of the first few values for each variable.
\end{itemize}

\begin{Shaded}
\begin{Highlighting}[]
\FunctionTok{glimpse}\NormalTok{(flights)}
\end{Highlighting}
\end{Shaded}

\begin{verbatim}
## Rows: 336,776
## Columns: 19
## $ year           <int> 2013, 2013, 2013, 2013, 2013, 2013, 2013, 2013, 2013, 2~
## $ month          <int> 1, 1, 1, 1, 1, 1, 1, 1, 1, 1, 1, 1, 1, 1, 1, 1, 1, 1, 1~
## $ day            <int> 1, 1, 1, 1, 1, 1, 1, 1, 1, 1, 1, 1, 1, 1, 1, 1, 1, 1, 1~
## $ dep_time       <int> 517, 533, 542, 544, 554, 554, 555, 557, 557, 558, 558, ~
## $ sched_dep_time <int> 515, 529, 540, 545, 600, 558, 600, 600, 600, 600, 600, ~
## $ dep_delay      <dbl> 2, 4, 2, -1, -6, -4, -5, -3, -3, -2, -2, -2, -2, -2, -1~
## $ arr_time       <int> 830, 850, 923, 1004, 812, 740, 913, 709, 838, 753, 849,~
## $ sched_arr_time <int> 819, 830, 850, 1022, 837, 728, 854, 723, 846, 745, 851,~
## $ arr_delay      <dbl> 11, 20, 33, -18, -25, 12, 19, -14, -8, 8, -2, -3, 7, -1~
## $ carrier        <chr> "UA", "UA", "AA", "B6", "DL", "UA", "B6", "EV", "B6", "~
## $ flight         <int> 1545, 1714, 1141, 725, 461, 1696, 507, 5708, 79, 301, 4~
## $ tailnum        <chr> "N14228", "N24211", "N619AA", "N804JB", "N668DN", "N394~
## $ origin         <chr> "EWR", "LGA", "JFK", "JFK", "LGA", "EWR", "EWR", "LGA",~
## $ dest           <chr> "IAH", "IAH", "MIA", "BQN", "ATL", "ORD", "FLL", "IAD",~
## $ air_time       <dbl> 227, 227, 160, 183, 116, 150, 158, 53, 140, 138, 149, 1~
## $ distance       <dbl> 1400, 1416, 1089, 1576, 762, 719, 1065, 229, 944, 733, ~
## $ hour           <dbl> 5, 5, 5, 5, 6, 5, 6, 6, 6, 6, 6, 6, 6, 6, 6, 5, 6, 6, 6~
## $ minute         <dbl> 15, 29, 40, 45, 0, 58, 0, 0, 0, 0, 0, 0, 0, 0, 0, 59, 0~
## $ time_hour      <dttm> 2013-01-01 05:00:00, 2013-01-01 05:00:00, 2013-01-01 0~
\end{verbatim}

\paragraph{\texorpdfstring{\texttt{kable()}}{kable()}}\label{kable}

The \texttt{kable()} function from the \textbf{knitr} package creates a
nicely formatted table of the data.

\begin{Shaded}
\begin{Highlighting}[]
\FunctionTok{library}\NormalTok{(knitr)}
\FunctionTok{kable}\NormalTok{(airlines)}
\end{Highlighting}
\end{Shaded}

\begin{longtable}[]{@{}ll@{}}
\toprule\noalign{}
carrier & name \\
\midrule\noalign{}
\endhead
\bottomrule\noalign{}
\endlastfoot
9E & Endeavor Air Inc. \\
AA & American Airlines Inc. \\
AS & Alaska Airlines Inc. \\
B6 & JetBlue Airways \\
DL & Delta Air Lines Inc. \\
EV & ExpressJet Airlines Inc. \\
F9 & Frontier Airlines Inc. \\
FL & AirTran Airways Corporation \\
HA & Hawaiian Airlines Inc. \\
MQ & Envoy Air \\
OO & SkyWest Airlines Inc. \\
UA & United Air Lines Inc. \\
US & US Airways Inc. \\
VX & Virgin America \\
WN & Southwest Airlines Co. \\
YV & Mesa Airlines Inc. \\
\end{longtable}

\paragraph{\texorpdfstring{\texttt{\$}
operator}{\$ operator}}\label{operator}

The \$ operator is a quick way to extract a single column from a data
frame.

\begin{Shaded}
\begin{Highlighting}[]
\NormalTok{airlines}\SpecialCharTok{$}\NormalTok{name}
\end{Highlighting}
\end{Shaded}

\begin{verbatim}
##  [1] "Endeavor Air Inc."           "American Airlines Inc."     
##  [3] "Alaska Airlines Inc."        "JetBlue Airways"            
##  [5] "Delta Air Lines Inc."        "ExpressJet Airlines Inc."   
##  [7] "Frontier Airlines Inc."      "AirTran Airways Corporation"
##  [9] "Hawaiian Airlines Inc."      "Envoy Air"                  
## [11] "SkyWest Airlines Inc."       "United Air Lines Inc."      
## [13] "US Airways Inc."             "Virgin America"             
## [15] "Southwest Airlines Co."      "Mesa Airlines Inc."
\end{verbatim}

We will more often explore data frames using \texttt{View()} and
\texttt{glimpse()}.

\paragraph{Types of Variables}\label{types-of-variables}

Data frames can contain different types of variables, each serving a
specific purpose. Examples include:

\begin{itemize}
\item
  \textbf{Identification variables} uniquely identify each observation
  (e.g., student ID, product serial number).
\item
  \textbf{Measurement variables} record numeric values or quantities
  (e.g., height, exam score).
\item
  \textbf{Categorical variables} group observations into categories
  (e.g., eye color, blood type).
\end{itemize}

\begin{Shaded}
\begin{Highlighting}[]
\FunctionTok{glimpse}\NormalTok{(airports)}
\end{Highlighting}
\end{Shaded}

\begin{verbatim}
## Rows: 1,458
## Columns: 8
## $ faa   <chr> "04G", "06A", "06C", "06N", "09J", "0A9", "0G6", "0G7", "0P2", "~
## $ name  <chr> "Lansdowne Airport", "Moton Field Municipal Airport", "Schaumbur~
## $ lat   <dbl> 41.13047, 32.46057, 41.98934, 41.43191, 31.07447, 36.37122, 41.4~
## $ lon   <dbl> -80.61958, -85.68003, -88.10124, -74.39156, -81.42778, -82.17342~
## $ alt   <dbl> 1044, 264, 801, 523, 11, 1593, 730, 492, 1000, 108, 409, 875, 10~
## $ tz    <dbl> -5, -6, -6, -5, -5, -5, -5, -5, -5, -8, -5, -6, -5, -5, -5, -5, ~
## $ dst   <chr> "A", "A", "A", "A", "A", "A", "A", "A", "U", "A", "A", "U", "A",~
## $ tzone <chr> "America/New_York", "America/Chicago", "America/Chicago", "Ameri~
\end{verbatim}

\paragraph{Help files}\label{help-files}

R provides built-in help files with documentation for functions,
datasets, and other objects. You can access a help file by typing
\texttt{?} followed by the name of a function or data frame in the
\textbf{Console}.

\begin{Shaded}
\begin{Highlighting}[]
\NormalTok{?flights}
\end{Highlighting}
\end{Shaded}

\begin{verbatim}
## starting httpd help server ... done
\end{verbatim}

Whenever you have questions about a function or dataset, make it a habit
to check the help file first.

\paragraph{Exercise 3}\label{exercise-3}

Using the \texttt{airports} dataset:

\begin{enumerate}
\def\labelenumi{\arabic{enumi}.}
\item
  What do the variables \texttt{lat}, \texttt{lon}, \texttt{alt},
  \texttt{tz}, \texttt{dst}, and \texttt{tzone} describe? (Hint: check
  the help file for \texttt{airports}.)
\item
  Name at least three variables where one is an identification variable
  and the other two are measurement variable.
\item
  Extract those three variables from the data frame.
\end{enumerate}

\paragraph{Other Data Structures}\label{other-data-structures}

R also includes other data structures such as \textbf{lists},
\textbf{matrices}, and \textbf{arrays}.

We will not cover these as much in this course, but you can read more
about them in \emph{Hands-On Programming with R}, Sections 5.3, 5.4, and
5.7.

\url{https://rstudio-education.github.io/hopr/r-objects.html}.

\end{document}
